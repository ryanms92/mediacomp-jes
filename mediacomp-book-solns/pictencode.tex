\chapter{Modifying Pictures using Loops}

% pictencode.tex, chapter of "Introduction to Media Computation"
\begin{exercises}

\begin{ex}
\begin{itemize}

\item We don't see red green and blue spots in pictures becuase a pixel is
so small that the human eye cannot detect it. However if the resolution (or
number of pixels per inch) were reduced -- for example, by zooming in on
the picture -- then you would be able to see dots of colors in a picture.

\item Luminance is the eye's perception of the brightness of light and
color.

\item Each color - red,green,and blue - is represented with 8 bits in a
computer.  Since each bit in a computer can be either a 0 or 1 (two
options) there are $2^8$ or 256 different possible representations you can
get with 8 bits.  This translates to 256 different shades for each of red
green and blue.  Since 0 is one of those representations, the highest you
can go is 255.

\item Since each of RGB uses 8 bits to represent it, it requires 3 * 8 or 24
bits to represent a pixel's color in a computer. With 24 bits, there are
$2^24$ different colors that can be represented in a computer. Since this
number is much greater than the number of colors that can be perceived by
the human eye, this representation gives more than enough colors.
\end{itemize}

\end{ex}


\begin{ex}
Hello back to you!
Hello
Hello
Hello
Hello
6
Hello back to you!
\end{ex}

\begin{ex}

\begin{example}
def decreaseRed(picture):
		for p in getPixels(picture):
		value = getRed(p)
		setRed(p,value0.9)

def decreaseRed(picture):
		for p in getPixels(picture):
		value = getRed(p)
		setRed(p,value0.8)
\end{example}

\end{ex}



\begin{ex}

\begin{example}

def decreaseBlue(picture):
		for p in getPixels(picture):
		value = getBlue(p)
		setBlue(p,value0.5)

def decreaseGreen(picture):
		for p in getPixels(picture):
		value = getGreen(p)
		setGreen(p,value0.5)

\end{example}
\end{ex}

\begin{ex}
(Sort of a subjective question, really.)
\end{ex}

\begin{ex}
Since each component, red green and blue can take on values only from
0-255, when you increase one of these components, it overflows (i.e.
it becomes a number greater than 255) When this happens, to prevent an
error from occuring, the numbers wrap around and it starts over again
at 0. So setRed(p,256) would be equivalent to setRed(p,0), 257 goes to 1,
etc. Since low numbers represent very little of that component,
when it wraps around via the increase color function, you eventually get
back to the low numbers -- so that color has in effect been decreased.
\end{ex}

\begin{ex}
\begin{example}
def clearRed(picture):
    for p in getPixels(picture):
    setRed(p,0)

def clearGreen(picture):
    for p in getPixels(picture):
    setGreen(p,0)
\end{example}
\end{ex}

\begin{ex}
\begin{example}
def maxBlue(picture):
    for p in getPixels(picture):
    setBlue(p,255)
\end{example}
\end{ex}

\begin{ex}
\begin{example}
XXX: not right yet. Still needs to calculate the percent thing. Way to go,
Kelly.

def changeColor(picture,percent,color):
    for p in getPixels(picture):
        if color == 1:
            r = getRed(p)
            setRed(p,r + (rpercent))
        if color == 2:
            g = getGreen(p)
            setGreen(p, g + (gpercent))
        if color == 3:
            b = getBlue(p)
            setBlue(p, b + (bpercent))
\end{example}

Write a new version of the function \code{increaseRed} and
\code{decreaseRed} (and blue and green) called \code{changeColor}
that takes as input a picture \emph{and} an amount to increase or
decrease a color by \emph{and} a number 1 (for red), 2 (for
green), or 3 (for blue).  The amount will be a number between
$-.99$ and $.99$.
\end{ex}

\begin{ex}
Take a look and see for yourself!
\end{ex}

\end{exercises}
