% soundcreate.tex, chapter of "Introduction to Media Computation"
\chapter[Combining Sounds]{Making Sounds by Combining Pieces}

\begin{exercises}

\begin{ex}
\begin{example}
def echo(delay):
  f = pickAFile()

  s1 = makeSound(f)
  s2 = makeSound(f)
  s3 = makeSound(f)

  for index in range(delay * 2 + 1, getLength(s1)):

    echoSample = 0.6 * getSampleValueAt(s2, index - delay)
    echoSample = 0.3 * getSampleValueAt(s3, index - delay * 2)

    setSampleValueAt(s1, index, getSampleValueAt(s1, index) + echoSample)

  play(s1)
  return(s1)
\end{example}
\end{ex}

\begin{ex}
If the frequency is doubled, the sound progresses twice as fast. Therefore
the length is half of the original.
\end{ex}

\begin{ex}
\begin{example}
#sound is the original sound you want to copy
#startIndex is the starting point of what you want to copy
#endIndex is the ending point of what you want to copy
#rate is the rate at which you want to multiply the frequency, use negative
for backwards play
#target is the target sound you want to copy the samples on to
#targetOffset is the offset that you wish to copy the samples to in terms
of sample numbers

def copySound(sound, startIndex, endIndex, rate, target, targetOffset):

  targetPointer = 0
  sourcePointer = startIndex

  while (sourcePointer * rate < endIndex * rate): # as long as there are
samples left to be copied
    targetPointer = targetPointer + 1
    value = getSampleValueAt(sound, int(sourcePointer))
    setSampleValueAt(target, targetOffset + targetPointer, value)
    sourcePointer = sourcePointer + rate

  return targetPointer

def makeDJnoises(original):
  length = getLength(original) / 22050 + 1 # ensuring that our target sound
is long enough to handle the copying
  target = makeEmptySound(length)
  offset = 0

  offset = offset + copySound(original, 1, 44100, 1, target, offset) # copy
2 seconds in a normal fashion
  offset = offset + copySound(original, 44100, 88200, 6, target, offset) #
copy the next 2 seconds at 6 times the frequency
  offset = offset + copySound(original, 88200, 44100, -6, target, offset) #
copy the same 2 seconds at 6 times the frequency backwards
  offset = offset + copySound(original, 44100, 88200, 6, target, offset) #
same thing again
  offset = offset + copySound(original, 88200, 44100, -6, target, offset)
  offset = offset + copySound(original, 88201, 132300, 1, target, offset) #
copy the next 2 seconds at a normal rate

  play(target)
  print str(offset) + " samples copied"
\end{example}
\end{ex}

\begin{ex}
The recipe is set up such if you increase the frequency of a sound, it will
repeat the sound when you run out of samples to copy. This "reset" occurs
when the sourceIndex = 1 line is hit, which usually happens when the
sourceIndex just slightly goes over the edge of the sound. Therefore,
subtracting the length of the sound would accomplish pretty much the same
effect.

The sound will start over each time it runs out of samples to copy. If you
want it to stop once it reaches the end of the sound, you can simply put a
break statement instead of a sourceIndex = 1. After that for loop, then add
a for loop that will go through the rest of the samples and set them all to
0.
\end{ex}

\begin{ex}
\begin{example}
def shiftDur(filename, numSamples):
  source = makeSound(filename)
  target = makeSound(filename)
  
  for targetIndex in range(1, min(numSamples, getLength(target)) + 1):
    progress = (targetIndex + 0.0) / numSamples)
    value = getSampleValueAt(source, int(getLength(source)  progress))
    setSampleValueAt(target, targetIndex, value)
    sourceIndex = sourceIndex + factor

  for targetIndex in range(numSamples + 1, getLength(target) + 1):
    setSampleValueAt(target, targetIndex, 0)

  play(target)
  return target
\end{example}
\end{ex}

\begin{ex}
Change the \code{shift} function in \recref{rec:shift} to
\code{shiftFreq} which takes a frequency instead of a factor, then
plays the given sound at the desired frequency.
\end{ex}

\begin{ex}
Play around and have fun!
\end{ex}

\begin{ex}
Have fun! Use math.sin to generate samples.
\end{ex}
\end{exercises}

