\chapter{Styles of Programming}
\begin{exercises}

\begin{ex} %% 14.1
(a) Weight two blocks against two other blocks.  If each set of two weights
the same, you know that the heavy block is in the two that
aren't on the scale.  Then weight those two individually against eachother.
The heavier one will tip the scale.  If on the first weight,
one of the pairs of blocks is heavier than the other, the heavy block is in
that pair.  Weight those two against each other and the heavy
one will tip the scale.  In either case, you only need to use the scale
twice.
\newline
(b) This is a logarithmic time search similar to a binary search.
\end{ex}

\begin{ex} %% 14.2
\end{ex}

\begin{ex} %% 14.3
Recursive:
\begin{example}
def upDown(word):
  if len(word) == 1:
    print word
    return
  print word
  upDown(word[:-1])
  print word
\end{example}

Iterative:
\begin{example}
def upDown(word):
  for char in range(len(word)):
    print word[:len(word)-char]
  for char in range(2,len(word)+1):
    print word[:char]  
\end{example}
\end{ex}

\begin{ex} %% 14.4
\end{ex}

\begin{ex} %% 14.5
\end{ex}

\begin{ex} %% 14.6
An instance is a specific object of the class.  The class is the code that
the instance is created from.
Functions are globally accessible.  Methods can only be run from an
instance of the class it is contained in.
Objected-Oriented programming uses classes 
\end{ex}

\begin{ex} %% 14.7
\begin{example}
#Original Box class
class Box:
  def \_\_init\_\_(self):
    self.setDefaultColor()
    self.size=10
    self.position=(10,10)
  def setDefaultColor(self):
    self.color = red
  def draw(self,canvas):
    addRectFilled(canvas,self.position[0],self.position[1],self.size,self.size,self.color)

#add these methods
  def setColor(self,newColor):
    self.color = newColor
  def setSize(self,newSize):
    self.size = newSize
  def setPosition(self,newPosition):
    self.position = newPosition
# new init -- calling added methods
  def \_\_init\_\_(self):
    self.setDefaultColor()
    self.setSize(10)
    self.setPosition((10,10))
# new setDefaultColor which calls setColor
  def setDefaultColor():
    self.setColor(red)
\end{example}
\end{ex}

\begin{ex} %% 14.8
\begin{example}
# Two more methods to add to class Box
def move(self,distance):
  #adding old position plus new distance to get new position
  newX = self.position[0] + distance[0]
  newY = self.position[1] + distance[1]
  self.setPosition = (newX,newY)
def grow(self,growth):
  newSize = self.size + growth
  self.setSize(newSize)
\end{example}
\end{ex}
\end{exercises}
