\chapter{Encoding, Creating, and Manipulating Files}

% files.tex, chapter of "Introduction to Media Computation"



    %\graphicspath{{figs/}{figs/files/}}



This chapter will eventually talk about directories and manipulating directories, and how to read and write files.



\section{How to walk through a directory}





To manipulate a directory of files, you have to \newcode{import} a \newterm{module} that contains additional functionality.  To manipulate the functions in the module, you'll have to use \newterm{dot notation}.



\begin{example}

>>> print file

/Users/guzdial/Work/mediasources/dark-bladerunner/dark-bladerunner 001.jpg

>>> import os

>>> for file in os.listdir("/Users/guzdial/Work/mediasources/dark-bladerunner"):

...    print file

...

dark-bladerunner 001.jpg

dark-bladerunner 002.jpg

dark-bladerunner 003.jpg

\ldots

\end{example}







\section{Copying files}



    \begin{recipe}{Simplest file copy}

\begin{example}

def copyFile(path1,path2):



  inFile  = open(path1,'r')

  outFile = open(path2,'w')



    temp = inFile.read()

    outFile.write(temp)



  outFile.close()

  return

\end{example}

\end{recipe}



What if we can't fit the whole file into memory?  But what if we run out of memory?



\begin{recipe}{Memory-protecting file copy}

\begin{example}

def copyFile(path1,path2):



  inFile  = open(path1,'r')

  outFile = open(path2,'w')



  more = 1                    # set so we can perform the first read

  while (more > 0 ):

                      # Alternative (less readable but cooler): while (more):



    temp = inFile.read(4096)  # reads 4096 bytes or whatever is left at end

    more = len(temp)          # when no more data more is set to 0

    outFile.write(temp)



  outFile.close()

  return

\end{example}

\end{recipe}



\code{temp} in these examples is a string.  If we want to copy \newterm{binary files}, we use this form.



\begin{recipe}{Memory-protecting binary file copy}

\begin{example}

def copyFile(path1,path2):



  inFile  = open(path1,'rb')

  outFile = open(path2,'wb')



  more = 1                    # set so we can perform the first read

  while (more > 0 ):

                      # Alternative (less readable but cooler): while (more):



    temp = inFile.read(4096)  # reads 4096 bytes or whatever is left at end

    more = len(temp)          # when no more data more is set to 0

    outFile.write(temp)



  outFile.close()

  return

\end{example}

\end{recipe}

