% functional.tex, chapter of "Introduction to Media Computation"

	%\graphicspath{{figs/}{figs/functional/}}



\section{Using \newcode{map}}

	\newcode{map} applies a function to a sequence of data.

Imagine that we have a function \code{increaseOneSample} that increases the amplitude of a \emph{single} sample.
\begin{example}
def increaseOneSample(sample):
  setSample(sample,getSample(sample)*2)

\end{example}

We can use that function and \newterm{apply} it to the whole sequence of samples like this:
\begin{example}
>>> file="/Users/guzdial/mediasources/hello.wav"
>>> sound=makeSound(file)
>>> result=map(increaseOneSample,getSamples(sound))
>>> play(sound)
\end{example}

But it turns out that we don't even have to create that extra function.  \newcode{lambda} allows us to create a function without even naming it!

\begin{example}
>>> file="/Users/guzdial/mediasources/hello.wav"
>>> sound=makeSound(file)
>>> result=map(lambda s:setSample(s,getSample(s)*2),getSamples(sound))
>>> play(sound)
\end{example}





	

